% LaTeX Article Template - customizing header and footer
\documentclass{article}

\newtheorem{thm}{Theorem}

% Set left margin - The default is 1 inch, so the following 
% command sets a 1.25-inch left margin.
\setlength{\oddsidemargin}{0.25in}

% Set width of the text - What is left will be the right margin.
% In this case, right margin is 8.5in - 1.25in - 6in = 1.25in.
\setlength{\textwidth}{6in}

% Set top margin - The default is 1 inch, so the following 
% command sets a 0.75-inch top margin.
\setlength{\topmargin}{-0.25in}

% Set height of the header
\setlength{\headheight}{0.3in}

% Set vertical distance between the header and the text
\setlength{\headsep}{0.2in}

% Set height of the text
\setlength{\textheight}{9in}

% Set vertical distance between the text and the
% bottom of footer
\setlength{\footskip}{0.1in}

% Set the beginning of a LaTeX document
\usepackage{multirow}

\usepackage{tikz}

\usepackage{fullpage}
\usepackage{graphicx}
\usepackage{amsthm}
\usepackage{url}
\usepackage{amssymb}
\usepackage{amssymb}
\usepackage{algpseudocode}
\graphicspath{%
    {converted_graphics/}% inserted by PCTeX
    {/}% inserted by PCTeX
}
%%%%%%%%%%%%%%%%%%%%%%%%%%%%%




\begin{document}\title{Homework $4$\\ Computer Science \\ B351 Spring 2017\\ Prof. M.M. Dalkilic}         % Enter your title between curly braces
\author{James Gregory}        % Enter your name between curly braces
\date{\today}          % Enter your date or \today between curly braces
\maketitle


% Redefine "plain" pagestyle
\makeatother     % `@' is restored as a "non-letter" character




% Set to use the "plain" pagestyle
\pagestyle{plain}
All work herein is mine
\section*{Introduction}
The aim of this homework is to get you well-acquainted with FOL and refutation.  You will turn-in one file, a *pdf with the written answers called \texttt{h4.pdf}.   I am providing this \LaTeX{} document for you to freely use as well.  Please enjoy this homework and ask yourself what interests you and then how can you add that interest to it!  All problems are worth 100 pts. each.  Include thet statement, ``All the work herein is mine.''
\newpage
\section*{Homework Questions}
\begin{enumerate}
\item Convert the following logical sentences to clausal form:
\begin{enumerate}
\item $\exists\ y\ p(y) \vee [\exists\ y\ (q(y) \rightarrow (\exists\ x\ (p(x) \vee\ q(x,y,C)))]$
\begin{quote}
$[[p(bob()), \neg q(y), p(steve()), q(steve(), y, C)]]$
\end{quote}
\item $\forall x \forall y \forall x\ d(x,y) \wedge d(y,z) \rightarrow d(x,z)$
\item $( P \vee Q) \wedge (\neg P \rightarrow (Q \vee R))$
\begin{quote}
$[[P, Q]]$
\end{quote}
\end{enumerate}
\item Let $\mathcal{U} = \{1,2,3\}$, $p = \{1,3\}$, $m = \{(1,1),(2,1),(3,2)\}$
\begin{enumerate}
\item Determine $\models \forall\ x\ \exists\ y\ m(x,y)$
\begin{quote}
True
\end{quote}
\item Determine $\models \forall\ y\ \exists\ x\ m(x,y)$
\begin{quote}
False
\end{quote}
\item Determine $\models \forall\ x\ \exists\ x\ m(x,x)$
\begin{quote}
True
\end{quote}
\item Determine $\models \exists \ x\ \forall\ y\ m(x,y)$
\begin{quote}
False
\end{quote}
\item Determine $\models \exists \ x\ \forall\ y\ m(y,x)$
\begin{quote}
True
\end{quote}
\item Determine $\models \exists \ x\ \forall\ x\ m(x,x)$
\begin{quote}
True
\end{quote}
\end{enumerate}
\item You've decided to add a new quantifier: $M$ that takes one variable.  The syntax is $M\ x\ f(x)$ for some sentence $f$.  The meaning of $M\ x\ f(x)$ is that the number of times $\sigma(u,x) f(x)$ is true, where $\sigma(u,x)$ is substituting a value from the domain $u \in \mathcal{U}$ is at least 1.5 times more than when it is false.  We can assume $\mathcal{U}$ is finite too.  Use the model in the previous problem.
\begin{enumerate}
\item Determine $\models M x\ p(x)$
\begin{quote}
True
\end{quote}
\item Determine $\models \forall \ x\ M y \ m(y,x) \rightarrow p(x)$
\begin{quote}
True
\end{quote}
\end{enumerate}
\item Ursala, Kaiser, and Shilah are dogs.  We know the following:
\begin{enumerate}
\item Ursala is silver.
\item Shilah is gray and loves Kaiser.
\item Kaiser is either gray or silver (but not both) and loves Ursala.
\end{enumerate}
What does this sentence mean? $\exists x \exists y (gray(x) \wedge silver(y) \wedge loves(x,y)$.  Use resolution refutation to prove this.
\begin{quote}
At least one grey dog loves a silver dog.
\end{quote}
\item Consider a robot that works in a mine -- it has to push some objects and not push others depending on a colored tag that is either green or red.  Here are the facts:
\begin{itemize}
\item If pushable objects are green, the non-pushable are red.
\begin{quote}
FOL: $green(p(x)) \rightarrow red(\neg p(y))$ \newline
Python: $[[\neg green(p(x)), red(\neg p(y))]]$
\end{quote}
\item All objects are either green or red.
\begin{quote}
FOL: $green(x) \vee red(x)$ \newline
Python: $[[green(x), red(x)]]$
\end{quote}
\item If there is a non-pushable object, then all pushable objects are green.
\begin{quote}
FOL:  $\neg p(bob()) \rightarrow (p(x) \wedge green(x))$ \newline
Python:  $[[p(bob()), p(x)], [p(bob()), green(x)]]$
\end{quote}
\item Object 1, a cart, is pushable.
\begin{quote}
FOL: $p(Cart)$ \newline
Python: $[p(Cart)]$
\end{quote}
\item Object 2, a pile of ore, is not pushable. 
\begin{quote}
FOL:  $\neg p(Ore)$ \newline
Python: $[\neg p(Ore)]$
\end{quote}
\end{itemize}
Assume you're trying to prove that there is a red object. 
\begin{itemize}
\item Rewrite the statements into FOL (formal) and show their robotic equivalent (Python).
\item Convert to clausal form.
\item Use refutation to prove the there is a red object, by working {\it only} on the robotic equivalent.  Clearly indicate the process.
\end{itemize}
\item Assume $\mathcal{U} = \{Alex, Bob, Cathy\}$, $M(x)$ means $x$ is a mechanic, $N(x)$ means $x$ works at NASA, $W(x,y)$ means $x$ worked with $y$, $I(x,y,z)$ means $x$ introduced $y$ to $z$.  Write constants $A,B,C$ to mean $Alex, Bob, Cathy$, respectively. Write the following in FOL:
\begin{itemize}
\item Cathy is a mechanic. Example:  $M(C)$
\item Bob is not a mechanic.
\begin{quote}
$\neg M(B)$
\end{quote}
\item Either Alex is a mechanic or Bob is, but I know Cathy works at NASA.
\begin{quote}
$(M(A) \vee M(B)) \wedge N(C)$
\end{quote}
\item Bob introduced Alex to Cathy, since Cathy works at NASA.
\begin{quote}
$N(C) \rightarrow I(B, A, C)$
\end{quote}
\item Someone is a mechanic, but everyone works at NASA.
\begin{quote}
$\exists\ x\ M(x) \wedge \forall\ y\ N(y)$
\end{quote}
\item Bob introduced himself to Cathy.
\begin{quote}
$I(B, B, C)$
\end{quote}
\item Nobody as been introduced to Alex.
\begin{quote}
$\forall\ x\ \forall\ y\ \neg I(x, y, A)$
\end{quote}
\item If someone introduced Bob to Alex, then Bob isn't a mechanic.
\begin{quote}
$\exists\ x\ I(x, B, A) \rightarrow \neg M(B)$
\end{quote}
\item Nobody works with anyone here.
\begin{quote}
$\forall\ x\ \forall\ y\ \neg W(x, y)$
\end{quote}
\item Somebody works with Cathy, but it's not a mechanic, because Cathy works at NASA.
\begin{quote}
$N(C) \rightarrow \exists\ x\ (W(x, C) \wedge \neg M(x))$
\end{quote}
\end{itemize}
\end{enumerate}
\end{document}
